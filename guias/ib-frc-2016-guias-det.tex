\documentclass[11pt]{article}
\usepackage[spanish]{babel} 
\usepackage[utf8]{inputenc}
\usepackage{amsmath}
\usepackage{color}
\renewcommand{\familydefault}{\sfdefault}

\textwidth 17cm
\textheight 25.8cm
\voffset -3.0cm
\hoffset -2.0cm

\begin{document}
\pagestyle{empty}
{Diciembre 2016}
\begin{center}
	{\Large \bf Física de Rayos Cósmicos} \\
{ \large \bf Rayos Cósmicos} \\ {\bf 1er semestre 2016}
\end{center}

\setcounter{enumi}{0}      %% Offset en numero de problema
\begin{enumerate}
	\item Imagine que se desea construir un detector Cherenkov, pero en lugar
		de usar agua, se usará disulfuro de carbono (CS$_2$), que tiene un
		índice de refracción mayor que el agua y que podemos suponer constante,
		i.e., $n_{\mathrm{CS}_2}(\lambda) \equiv n = 1.627$, y con una densidad
		algo mayor también, $\rho_{\mathrm{CS}_2}(\lambda) \equiv \rho =
		1.3$\,g\,cm$^{-3}$. Para construir al detector se utilizará una esfera
		de radio $r=1$\,m, y en algún punto de su superficie se montará un PMT
		con la siguiente eficiencia cuántica:
		$$
			\mathrm{QE} \equiv \frac{\mathrm{fotoelectrones\
			producidos}}{\mathrm{fotones\ incidentes}} = 
			\begin{cases}
				0    & \lambda < 250\mathrm{\ nm} \\
				0.30 & 250\mathrm{\ nm} \leq \lambda \leq 600\mathrm{\ nm} \\
				0    & \lambda > 600\mathrm{\ nm}.
			\end{cases}
		$$
		Utilizando como guía las expresiones calculadas en clase, y la curva
		del poder de frenado para electrones en CS$_2$ (ver tabla), calcule:
		\begin{enumerate}
			\item El ángulo máximo de emisión Cherenkov $\theta_{\mathrm{Ch}}$
				en este líquido. 
			\item El umbral de producción Cherenkov $\beta_{\mathrm{Ch}}$, y el
				correspondiente momentum $p$, energía cinética $K$ y energía
				total $E$ que deben tener electrones, muones y protones para
				ser detectados.  Luego, calcule la energía mínima que debe
				tener un fotón para ser detectado mediante el proceso de
				creación de pares en el CS$_2$.
			\item A partir del rango estimado para electrones (ver tabla),
				calcule, cuando corresponda, el número total de fotones
				Cherenkov producidos por la propagación de un electrón con
				energía $E=\{0.5; 5; 50; 500\}$\,MeV (por simplicidad, suponga
				que la curva de producción de fotones es una función escalón,
				que vale 0 por debajo de la energía umbral de producción y el
				valor de saturación por encima. Haga y describa las
				aproximaciones que considere necesarias para estimar el total
				de fotones.)
		\end{enumerate}
		\begin{center}
			\begin{tabular}{|c|c|c|c|c|}
				\hline
				\multicolumn{5}{c}{ESTAR: Stopping Powers and Range Tables for
				Electrons} \\
				\multicolumn{5}{c}{Carbon disulfide
				$\rho=1.2927$\,g\,cm$^{-3}$, Ionization=175.9\,eV} \\
				\hline
				$K$ (MeV) & $S_{\mathrm{Col}}$ (MeV cm$^2$/g) &
				$S_{\mathrm{Rad}}$ (MeV cm$^2$/g) & $S_{\mathrm{Tot}}$ (MeV
				cm$^2$/g) & Range (g/cm$^2$)\\
				\hline
				5.000E-01 & 1.653E+00 & 1.425E-02 & 1.668E+00 & 2.186E-01 \\ 
				5.000E+00 & 1.635E+00 & 1.440E-01 & 1.779E+00 & 2.961E+00 \\ 
				5.000E+01 & 1.900E+00 & 1.995E+00 & 3.895E+00 & 1.942E+01 \\ 
				5.000E+02 & 2.093E+00 & 2.255E+01 & 2.465E+01 & 5.958E+01 \\
				\hline
			\end{tabular}
		\end{center}
\end{enumerate}
\end{document}
