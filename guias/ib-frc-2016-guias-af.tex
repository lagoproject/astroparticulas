\documentclass[11pt]{article}
\usepackage[spanish]{babel} 
\usepackage[utf8]{inputenc}
\usepackage{amsmath}
\usepackage{amssymb}
\usepackage{color}
\renewcommand{\familydefault}{\sfdefault}

\textwidth 17cm
\textheight 25.8cm
\voffset -3.0cm
\hoffset -2.0cm

%\usepackage{pdfpages}

\begin{document}
\pagestyle{empty}
{10 Septiembre 2016}
\begin{center}
	{\Large \bf Física de Rayos Cósmicos} \\
{ \large \bf Rayos Cósmicos} \\ {\bf 1er semestre 2016}
\end{center}

\setcounter{enumi}{0}      %% Offset en numero de problema
\begin{enumerate}
	\item Suponiendo que el espectro de rayos cósmicos (RC) sigue una ley de
		potencias de la forma $j(E)=j_0 E^{\alpha}$, calcule el número total de
		rayos cósmicos que arriban a la Tierra por año y por km$^2$ en los
		siguientes rangos de energía:
		\begin{enumerate}
			\item $10^{7} \leq E/\mathrm{GeV} \leq 10^{8}$ con $\alpha=-3.3$.
			\item $10^{7} \leq E/\mathrm{GeV} \leq 10^{8}$ con $\alpha=-3.0$.
			\item $10^{3} \leq E/\mathrm{GeV} \leq 10^{4}$ con $\alpha=-2.7$.
		\end{enumerate}
		en todos los casos obtenga el valor de $j_0$ de los espectros
		publicados (ver p. ej. espectro en U01C01).
	\item Verifique que la fuerza de Lorentz relativista puede escribirse en
		forma covariante como $$ \frac{dp^\mu}{d\tau} = q F^{\mu\nu} u_\nu$$
		donde $p^\mu$ es el 4-momento, $p^\mu = (\gamma m c, p_x, p_y, p_z)$,
		$\tau$ es el tiempo propio de la partícula, $F^{\mu\nu}$ es la forma
		contravariante del tensor de Maxwell y $u_\nu$ es la forma covariante
		de la 4-velocidad, $u_\nu= \gamma (c, -v_x, -v_y, -v_z)$. Notar que se
		usó la métrica usual en partículas, $\eta=\mathrm{diag}(1,-1,-1,-1)$.
	\item Demuestre que el radio de Larmor de una partícula de masa $m$ y
		carga $q$ que se mueve en presencia de un campo magnético $\vec B$ con
		velocidad $\vec v$ formando un ángulo $\theta$ con el campo magnético
		puede escribirse como $$ r= \frac{\gamma m v \sin \theta}{|q|
		B}.$$Luego, haciendo los cambios de unidades que considere necesarios,
		pruebe que la expresión anterior puede reescribirse como $$r = 3.3 
			\left ( \frac{\gamma m c^2}{\mathrm{GeV}} \right )
			\left ( \frac{v_\perp}{c} \right )
			\left ( \frac{e}{|q|} \right )
			\mathrm{\ \ metros}.
		$$
	\item Usando las variables de Mandelstam, y en particular
		$s=E_{\mathrm{CM}}^2$, verifique que la energía de la colisión en el
		LHC ($13$\,TeV) es igual a $\sim 10^{5}$\,TeV en el sistema de
		laboratorio (una de las partículas está en reposo, aire).
	\item Suponiendo que la capacidad de una fuente le permite acelerar
		protones hasta una energía de corte $E_c = 4\times10^{15}$\,eV.
		Calcule el espectro combinado (H,He,C,Fe) de la fuente suponiendo que
		el flujo de 1-Hidrógeno es $\mathcal{F}_{\mathrm{H}} = (1.15\times10^{-5})
		E^{-2.77}\,\mathrm{m}^{-2}\,\mathrm{sr}^{-1}\,\mathrm{s}^{-1}\,\mathrm{TeV}^{-1}$,
		el flujo de 4-Helio es  $\mathcal{F}_{\mathrm{He}} = (7.19\times10^{-6})
		E^{-2.64}\,\mathrm{m}^{-2}\,\mathrm{sr}^{-1}\,\mathrm{s}^{-1}\,\mathrm{TeV}^{-1}$,
		el flujo de 12-Carbono es $\mathcal{F}_{\mathrm{C}} = (1.06\times10^{-6})
		E^{-2.66}\,\mathrm{m}^{-2}\,\mathrm{sr}^{-1}\,\mathrm{s}^{-1}\,\mathrm{TeV}^{-1}$, 
		y el flujo de 56-Hierro es $\mathcal{F}_{\mathrm{Fe}} = (1.78\times10^{-6})
		E^{-2.6}\,\mathrm{m}^{-2}\,\mathrm{sr}^{-1}\,\mathrm{s}^{-1}\,\mathrm{TeV}^{-1}$.
	\item Siguiendo los lineamientos de Protheroe\&Clay, 2004, verifique que
		en el mecanismo de Fermi de 2do orden predice un incremento medio de
		energía $\langle \Delta E \rangle \simeq 4/3 \beta^2 E$ y un espectro
		del tipo ley de potencias $J(E) \propto E^{\alpha}$ con $\alpha < -1$.
		Luego describa los principales inconvenientes de este modelo. Repita
		lo anterior para el caso del mecanismo de Fermi de primer orden
		($\langle \Delta E \rangle \simeq 4/3 \beta E$ y $\alpha \simeq -2$).
	\item Usando el invariante de Mandelstam $s$, verifique los umbrales de energía de los siguientes procesos:
		\begin{description}
			\item[Fotoproducción de piones] $p^+ + \gamma_{\mathrm{CMB}} \to p^+ + \pi^0$, $E_{p^+} \gtrsim 30$\,EeV
			\item[Fotonucleoproducción de piones] $A + \gamma_{\mathrm{CMB}} \to A + \pi^0$, $E_{A} \gtrsim 30 \left ( 1 + m_\pi / \left ( A m_p \right ) \right )$\,EeV. 
			\item[Fotoproducción de pares] $p^+ + \gamma_{\mathrm{CMB}} \to p^+ + e^+ + e^-$, $E_{p^+} \gtrsim 3$\,EeV
			\item[Fotonucleoproducción de piones] $A + \gamma_{\mathrm{CMB}} \to A + e^+ + e^-$, $E_{A} \gtrsim 3 \left ( 1 + m_e / \left ( A m_p \right ) \right )$\,EeV.
		\end{description}
\end{enumerate}


\end{document}
