\documentclass[11pt]{article}
\usepackage[spanish]{babel} 
\usepackage[utf8]{inputenc}
\usepackage{amsmath}
\usepackage{color}
\renewcommand{\familydefault}{\sfdefault}

\textwidth 17cm
\textheight 25.8cm
\voffset -3.0cm
\hoffset -2.0cm

\begin{document}
\pagestyle{empty}
{10 Septiembre 2016}
\begin{center}
	{\Large \bf Física de Rayos Cósmicos} \\
{ \large \bf Rayos Cósmicos} \\ {\bf 1er semestre 2016}
\end{center}

\setcounter{enumi}{0}      %% Offset en numero de problema
\begin{enumerate}
	\item Un modelo más realista para el modelo de Heitler implica suponer que
		luego de cada capa atmosférica, cuyo espesor es
		$X_{EM}=37.1$\,g\,cm$^{-2}$, cada partícula produce $g_{\mathrm{EM}}$
		nuevas partículas. De esta forma, el modelo original de Heitler supone
		$g_{EM}=2$. Obtenga según este modelo realista los nuevos parámetros de
		la cascada, $N_{\max}$, $X_{\max}$ y $\Lambda$ ({\it{elongation
		rate}}), como función de $g_{EM}$. Luego, evalúe esos parámetros para
		los casos $g_{EM}=2, 6, 13, 20$.
	\item La atmósfera marciana es una mezcla de gases con la siguiente
		composición: $96\%$ de CO$_2$, $2\%$ de Ar, $1.8\%$ de N$_2$ y $0.2\%$
		de O$_2$. Es tan tenue que la presión atmosférica en su superficie es
		de sólo $0.6$\,hPa (como referencia, la presión atmosférica en la
		superficie terrestre es $1013.2$\,hPa). Repita los cálculos del punto
		anterior para el caso de Marte y compare los resultados obtenidos con
		los de la Tierra. Ayuda: algunos datos adicionales que podrían llegar a
		usar: masa de Marte: $6.42\times10^{23}$\,kg; radio de Marte:
		$3400$\,km. El número másico de una mezcla calcula simplemente como un
		promedio pesado por la fracción de cada constituyente, $\langle A
		\rangle = \sum_i A_i x_i$. Para el número atómico de la mezcla debe
		calcularse el número atómico efectivo,
		$Z_{\mathrm{eff}}=\sqrt[2.94]{\sum_i f_i Z_i^{2.94}}$, donde $Z_i$ es
		el número atómico del elemento $i$-ésimo y $f_i$ es la fracción de
		carga de cada elemento (es decir, $f_i=Z_i/\sum_i Z_i$). Por ejemplo,
		para el caso del agua, H$_2$O, $Z_{\mathrm{eff}}=7.42$.
	\item El modelo de Glasmaher-Matthews para cascadas hadrónicas es un modelo
		simple que permite comprender la evolución de una cascada iniciada por
		un hadrón. Hemos visto en clase el caso de una lluvia atmosférica
		extendida (EAS) iniciada por un protón de energía $E_p$, donde
		obtuvimos que la mayor parte de la energía se encuentra en el canal
		electromagnético ($E_{\mathrm{EM}} = 1 - (E_p/E_\pi)^{\beta_\pi - 1}
		\simeq 0.9$ para $\beta_\pi=0.85$). Luego, es válido suponer que a
		primer orden $X_{\max} \simeq X_{\max}^{\mathrm{EM}}$, donde este último
		término corresponde a la posición del máximo de una cascada equivalente
		pero iniciada por un fotón de energía $E_\gamma=E_p / (3
		N_{\mathrm{CH}})$ (esto surge de suponer que en la primer interacción
		se producen $N_{\mathrm{CH}}$ y $N_{\mathrm{CH}}/2$, que decaen
		inmediatamente según la reacción $\pi^0 \to 2\gamma$, y luego
		$N_\gamma=N_{\mathrm{CH}}$). Bajo esta aproximación, demuestre que la
		posición del máximo para una cascada iniciada por un protón puede
		aproximarse cómo $X_{\max}^p = X_0 + X_{\max}^{\mathrm{EM}} -
		126$\,g\,cm$^{-2}$ para $N_{\mathrm{CH}}=10$ y $X_0$ corresponde al
		punto de primera interacción.
	\item El modelo de superposición para una EAS iniciada por un hadrón de
		masa $A$ y energía $E_p$ sostiene que la cascada resultante es
		equivalente a $A$ cascadas simultáneas iniciadas por sendos protones de
		energía $E_p/A$.  Este modelo se soporta en el hecho de que a las
		energías más altas, $E_p \gg m_p$, la energía de ligadura por nucleón
		(típicamente $B/A\simeq 8.8$\,MeV) es despreciable frente a $E_p/A$.
		Utilice entonces el modelo de Glasmaher-Matthews combinado con el
		modelo de superposición para verificar que:
		\begin{enumerate}
			\item El número de muones de una lluvia iniciada por un núcleo de
				hierro ($^{56}$Fe$_{26}$) es $\simeq 50\%$ mayor que el número de
				muones de una lluvia iniciada por un protón (estrictamente,
				$N_\mu^A \simeq A^{1-\beta_\pi} N_\mu^p$).
			\item La posición del máximo puede aproximarse como $X_{\max}^A =
				X_{\max}^{p} - X_{\mathrm{EM}} \ln A$.
			\item Las fluctuaciones en la posición del máximo de distintas
				lluvias con la misma energía del primario son menores para los
				hierros que para los protones. 
		\end{enumerate}
\end{enumerate}


\end{document}
